\chapter{Conclusion}
% Konkluder ved hjælp af studieordningen og vores punkter under problem formuleringen.

%Start med at forklare projektet - Vi har lavet dette sprog som skla bruges af disse mennesker og som kan .. bum bum bum.. 
%Syntax, symantic osv. 

%rund af med at hvad intentionen var med sporget, og hvor virkeligheden så er endt. 
%Skal hænge sammen med abstract, (som indeholder hvad raporten indeholder..  fx. så mange typer og så mange test osv, vi har defineret en brugertest, men vi nåede de her tests.. 

This report details the development of \lang a programming language aimed towards introducing beginner programmers to text-based programming and some of the concepts of the industry standards, in an educational format. \lang has fundamental features, such as basic arithmetic, control structures, abstraction through functions, etc. with high readability. The challenge of designing \lang was a balancing act between using readable concepts from block-based languages, whilst still designing features that introduce users to the industry standards for programming languages. Based on this initial problem, the following problem statement was defined. 

\begin{center}
    \textit{How can a text-based programming language be developed for beginner programmers, using readable concepts from block-based languages, while focusing on facilitating the transition to programming languages used in the industry?}
\end{center} 

Based on the requirements described in the problem statement section \ref{problemstatement}, which aims towards creating a successful compiler for \lang, it can be concluded, that the parts and phases of a compiler that are required have been successfully implemented, tested and described. This includes but is not limited to the creation of a lexer, parser, AST design, correct scope rules, successful type checking, symbol table creation, code generation, testing as well as defining operational semantics for \lang. \\

Finally, It can be concluded that \lang has successfully met most requirements defined in section \ref{requirements} table \ref{tab:requirements}. Unfortunately, because of the lack of user testing, \lang's ability to serve as a beginner-friendly language cannot be entirely concluded as successful in a practical manner. Although the requirements are built upon the analysis, it is difficult to disregard the bias of an experienced programmer, compared to the mind of a beginner programmer. However, the analysis provides a blueprint of how \lang should be designed in order to cater to beginner programmers in high school. Hence it can be concluded that \lang has successfully fulfilled the problem statement.\\


%------------

% Based on the language requirements defined in table \ref{tab:requirements}, it can be concluded that the language has successfully met all the requirements, which were designed with the problem statement in mind.
% Unfortunately, because of the lack of user testing done on the language, its ability to serve as a beginner-friendly language cannot be entirely concluded as successful in a practical manner. However, the requirements are built upon Sebesta's language criteria. The analysis, therefore, provides a blueprint of how the language should be designed in order to cater to beginner programmers in high school. Hence it can be concluded that the language has successfully fulfilled the problem statement.In order to fully conclude whether the language has educational benefits, proper user testing should be done on the appropriate target audience. \\

% Finally, it can be concluded based on the requirements described in the problem statement section \ref{problemstatement}, which aim towards creating a successful compiler for \lang, that the parts and phases of a compiler that are required have been successfully implemented, tested and described. This includes but is not limited to a lexer, parser, AST design, correct scope rules, successful type checking, symbol table creation, code generation, testing as well as defining operational semantics for \lang. 


% \begin{itemize}
% %  \item Analyze beginner programmers.
%   \item Define language criteria.
% %  \item Explain the programming paradigm of \lang.
% %  \item Create a MoSCoW requirement table.
%   \item Create a Context-Free Grammar (CFG).
%   \item Define semantics.
%   \item Define the scope rules.
%   \item Design Abstract Syntax Tree (AST).
%   \item Create a lexer and parser.
%   \item Build symbol table and handle type checking.
%   \item Successful code generation
% %  \item Show how the Syntax analysis, the Contextual analysis, and the code generation have been implemented.
%   \item Testing of \lang, including unit testing, integration testing, and acceptance testing.
% %  \item Evidence in the form of tests, to support the claim that our language helps provide the beginner programmer with a better transition from our language to an industry-standard language.
% \end{itemize}