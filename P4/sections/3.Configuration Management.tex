\section{Configuration Management}
This section will cover how this project is managed, including the use of agile techniques, how the use of version control is conducted, and the tools used to manage the project. This section can be skipped, if there is no interest in the configuration management of this project. 

\subsection{Agile Project Management} \label{agileprojectmanagement}
Jira is a proprietary issue-tracking product that enables you to work more Scrum-like, compared to traditional project management approaches, like going over tasks in the waterfall method. It provides the opportunity to have Kanban boards and a road map, as well as a backlog. The backlog gives us the opportunity to write down tasks, also called issues in Jira. From the backlog, you are able to create sprints. The concept of sprints is well-known in the world of agile project management and consists of a dedicated period of time in which a set of tasks will need to be completed on a project \cite{AdobeProjectSprints}. These issues are then able to be assigned to specific sprints and which are then displayed on the Kanban board. So we have the ability to only display certain issues that are needed for these specific sprints. An example of how we used this was in the last month of the project when we wrote all of the issues that have to be handled before we submit the project. We then created sprints for every week until the submission date to make sure that we had sharp deadlines. In 3rd semester we worked on a Trello board and created cards to move around from "To Do" to "Review 1" and so on. Even though it did the job, it did not quite manage to solve our needs. It was a bit confusing having everything displayed at once on the to-do list. In this semester's project, we decided to use Jira instead \cite{jira}.

%This is a perfect solution for this group and is definitely a concept that we will continue to use, if we were to. 

\subsection{Version Control}
Just like the agile project management in section \ref{agileprojectmanagement}, we have a lot of experience with version control through GitHub \& Overleaf. Version control, also known as source control, is the practice of tracking and managing changes to software code \cite{whatisversioncontrol}. This worked well throughout the project and gave us the opportunity to have the project code neatly organized, as opposed to having different code projects locally on each of our computers. Version control through GitHub enables us to get a better understanding of the code others wrote. GitHub gives us the opportunity to put rules on branches. We put a rule on the main branch, that said you cannot merge from a branch into the main branch without having two additional group members to approve of the code. Since the project would end up with multiple group members working on it at the same time, it could quickly become entangled and complicated. Therefore we used GitHub, and create branches for each increment in the backlog, thus keeping the main branch protected. Furthermore, if anything goes wrong on the main branch, it is easily undone by rolling back to an older commit. All code that is displayed in this report, is placed inside a GitHub repository \cite{p4-project-rep}.\\

Another kind of version control we use is Overleaf. Overleaf is a collaborative cloud-based \LaTeX\ editor used for writing, editing, and publishing scientific documents \cite{overleafwikipedia}. We used \LaTeX\ \& Overleaf throughout their studies and it was therefore natural for us to write our report in Overleaf. Overleaf essentially does version control by itself. Every 5 minutes it makes a git commit with every file, and we are able to review changes since the beginning of the project.

\subsection{Supervisor meetings}
The supervisor meetings are weekly meetings that are either physical in the group room or online. Throughout the week between the meetings, we write everything down that we have questions for, and what we would like feedback on in the report by giving a reading guide, then 48 hours before the meeting an agenda, and these questions and the reading guide, and the report are sent to the supervisor.\\
At the supervisor meetings, one group member takes notes and another goes over the agenda to always have a structure. These roles switch every supervisor meeting so everyone gets to talk.

\subsection{Additional tools}
Some additional tools that we use are Discord, Outlook.com, and ChatGPT. 
\begin{itemize}
    \item All internal communication goes through a Discord server. This provides us with the ability to divide different sections into smaller chat channels to create a better overview.
    \item All external communication is handled through outlook.com, which includes supervisor emails, and study secretary.
    \item We used AI assistance in the development of the compiler, at no point is ChatGPT a co-writer in the report. We agreed on how the use of ChatGPT should be, and that it should only be to get some advice on specific questions and guidance.
\end{itemize}

\begin{comment}
   
\begin{itemize}
    \item Discord is a VoIP and instant messaging social platform. Users have the ability to communicate with voice calls, video calls, text messaging, media, and files in private chats or as part of communities called "servers" \cite{discordwikipedia}. All internal communication goes through the group's Discord server. This provides us with the ability to divide different sections into smaller chat channels to create a better overview. This is also used when the group works at home, as you are able to talk with each other through voice channels, just like e.g. Microsoft Teams.
    \item Outlook.com is a free webmail version of Microsoft Outlook, using a similar user interface \cite{outlookwikipedia}. 
    \item ChatGPT is an artificial intelligence (AI) chatbot developed by OpenAI and released in November 2022. The "Chat" in the name is a reference to it being a chatbot, and the "GPT" stands for generative pre-trained transformer—a type of large language model (LLM) \cite{chatgptwikipedia}. 
\end{itemize}
 
\end{comment}