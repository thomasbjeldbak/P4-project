\pdfbookmark[0]{English title page}{label:titlepage_en}
\aautitlepage{%
  \englishprojectinfo{
     PEAK+ - Programming Education and Knowledge+%title
  }{%
     Design, definition and implementation of programming languages %theme
  }{%
    Spring Semester 2023 %project period
  }{%
    SW4-01 % project group
  }{%
    %list of group members
    Charlotte Sundahl Elleby\\ 
    Christian Povlsen\\
    Nichlas Seerup Hjorth\\
    Rasmus Bartholomay Vikøren Borup\\
    Thomas Bjeldbak Madsen\\
    Thomas Ilum Andersen
  }{%
    %list of supervisors
    Lone Leth Thomsen 
  }{%
    1 % number of printed copies
  }{%
    \today % date of completion
  }%
}{%department and address
  \textbf{Department of Computer Science}\\
  Aalborg University\\
  \href{http://www.aau.dk}{http://www.aau.dk}
}{% the abstract
The learning curve of a new programming language for beginners who have experience with block-based languages is equally difficult for beginners, with the same amount of experience, but with text-based languages. This provides a need for a language that introduces beginners to the textual industry-standard programming languages. This report details the development and creation of \lang, a programming language that aims to fulfil this aforementioned need. The necessary syntax for \lang was developed through an analysis of existing programming languages, including Scratch, Quorum, C, and Python. The languages were analyzed with Sebesta's language criteria. Through this analysis, the syntax and semantics of \lang were designed. A functional compiler was built, by going through 3 essential phases: syntax analysis, contextual analysis, and code generation. ANTLR was used as a parser generator tool, and a visitor pattern, developed in C\# was used to traverse the compiler. The compiler was tested through unit, integration, and acceptance testing. The language was concluded to have theoretically solved the need for a language between block-based and text-based programming languages, however, to fully conclude a successful solution, proper user testing must be completed. 
}

\begin{comment}
\cleardoublepage
{\selectlanguage{danish}
\pdfbookmark[0]{Danish title page}{label:titlepage_da}
\aautitlepage{%
  \danishprojectinfo{
    Rapportens titel %title
  }{%
    Semestertema %theme
  }{%
    Efterårssemestret 2010 %project period
  }{%
    XXX % project group
  }{%
    %list of group members
    Forfatter 1\\ 
    Forfatter 2\\
    Forfatter 3
  }{%
    %list of supervisors
    Vejleder 1\\
    Vejleder 2
  }{%
    1 % number of printed copies
  }{%
    \today % date of completion
  }%
}{%department and address
  \textbf{Elektronik og IT}\\
  Aalborg Universitet\\
  \href{http://www.aau.dk}{http://www.aau.dk}
}{% the abstract
  Her er resuméet
}}
\end{comment}